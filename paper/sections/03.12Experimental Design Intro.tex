The game segment in our experiment consists of 3 parts, namely Baseline game, Anticipation game and Scarcity game, where the last two are the treatments in our experiment. The baseline game is the standard forestry extraction game following \citet{janssen2013breaking}. The anticipation game follows a similar protocol to that in the baseline adjusted to a setting where there are risks of climate shock to occur and destroy the existing shared resources. In the scarcity game, subjects were put in a situation where the resource stock available is reduced significantly.

 We employed between and within subject design in our experiment game. Our within subject design is through the linear successions of the game parts, and between game design is employed under Anticipation game only, where subjects were randomly assigned to high or low probability group treatments. We designed the game as such in order to observe any behavioural changes that occurs between Baseline against the Anticipation game. By comparing the observed behavior between the two, we can observe how subject's behaviour might change in the presence of exogenous risk of climate shock. This risk presents uncertainty to the subjects individually and as a group, which consequently induces a lower expected value of the shared resource, providing incentives for subjects to harvest more now rather than later. Such design for the first two parts of our game will allow us to test our first hypothesis.

\medskip
\begin{adjustbox}{minipage=.7\textwidth,margin=0pt \smallskipamount,center}
    \itshape \noindent \textbf{H1}: An increased risk of climate shocks, that induces a lower expected future value of the resource, leads to an increase in harvest rates and a decrease in the stock of the resource.
\end{adjustbox}
\medskip

 Under the Anticipation part of the game, we randomly assigned subjects into 2 climate shock risk group, namely high and low probability. We designed it as such in order to see whether the relative size of the risk to occur has significant impact on subject's CPRs harvesting behaviour. With both groups experiencing elevated risk of climate shock, we can thus compare between the groups for observed behaviour changes.

 At the end of Anticipation game, groups may or may not experience actual climate shock. This differing random outcome allow us to further compare the results of Scarcity game where all subject restart a game but with only half of the resources they previously began the game with. Such setting is useful to compare whether subjects who have had prior experience of scarcity in the Anticipation game will behave differently from subjects who did not experience such scarcity when they later play our Scarcity game. However, we will not discuss our findings from this part in detail and will keep this for future studies.

 The survey within our laboratory experiment contains questions asking the participants regarding the optimum solution to the game, we adopted also questions from the short 24-item version of the Environmental Attitudes Inventory \cite{milfont2007} and the Global Preference Survey \cite{falk2018global}. The last two adopted survey questions are then used to measure personal attitude and perception towards environmentalism, and their preferences with respect to altruism, reciprocity, time preference, risk preference and trust. All these variables we believe to be important exogenous factors influencing individual and group extraction behaviour in the experiment, which led us to our second hypothesis.

\medskip
\begin{adjustbox}{minipage=.7\textwidth,margin=0pt \smallskipamount,center}
    \itshape \noindent \textbf{H2}: Subjects categorized as pro-environmental according to the Environmental Attitudes Inventory (EAI) extract less of the resource.
\end{adjustbox}
\medskip
