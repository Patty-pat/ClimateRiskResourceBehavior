%\documentclass[cprs.tex]{subfiles}
Given the uncertainty of a shock to occur, individual and group's social payoff depends on the expected payoff given the treatments probability. This means that for our second game, individual expected payoff of those who were assigned to a high or low probability group can be expressed as
\begin{equation}
\label{eq:10}
    \pi_{i} = 2\sum_{t=1}^{5} (x_{i,t})+  Z_{5} \times 4 \left[ \dfrac{(0,5 \times p)+(1 \times 1-p)}{3} \right] \\
\end{equation}
And the group expected payoff adjusted to
\begin{equation}
\label{eq:11}
    \Pi_{group} = 2\sum_{t=1}^{5} (x_{i,t})+ Z_{5} \times 4 \left[ (0,5 \times p)+(1 \times 1-p) \right]
\end{equation}
Given the increase in uncertainty of their social payoff, it provides an incentive for participants to extract more earlier in the game to gain payoff for themselves now rather than later. This potential 50\% drop in social payoff reduces their expected future value.

The expected value of trees not harvested in the Anticipation game can be found by calculating the following:

\begin{table}[htbp!]
  \centering
  \caption{Expected Value of Trees at the End of the Game by High and Low Probability Group}\label{tab:7}
  \begin{tabular}{p{0.2\linewidth} | p{0.2\linewidth} | p{0.15\linewidth} | p{0.35\linewidth}}
    \toprule
    \textbf{Group Type} & \textbf{Probability of Shock Occurring} & \textbf{\% Trees Remain} & \textbf{Expected Total Value of Each Tree Not Harvested} \\ %\midrule
    \multirow{2}{*}{\textbf{High}} & P(Shock)=0.8 & 50\% & \multirow{2}{*}{$ 4 \left[0.8(0.5)+(1-0.8)(1)\right]=2.4$} \\ \cline{2-3}
     & P(No Shock)=(1-0.8) & 100\%  \\ %\midrule
    \multirow{2}{*}{\textbf{Low}} & P(Shock)=0.2 & 50\% & \multirow{2}{*}{$ 4 \left[0.2(0.5)+(1-0.2)(1)\right]=3.6$} \\ \cline{2-3}
    & P(No Shock)=(1-0.2) & 100\%  \\
    \bottomrule
  \end{tabular}
\end{table}


For comparison between Baseline, Anticipation and Scarcity Game, Table \ref{tab:4} summarizes this.
\begin{table}[htbp!]
  \centering
  \caption{Summary of Expected Value of Trees by Game}\label{tab:4}
  \begin{tabular}{p{0.5\linewidth}|p{0.2\linewidth}|p{0.2\linewidth}}
    \toprule
    \textbf{Game} & \textbf{Value of Tree Harvested} & \textbf{Value of Tree Not Harvested} \\ \hline \hline
    Baseline Game & \centering{2} & \centering{4} \\
    \midrule
    Anticipation Game High Probability Group & \centering{2} & \centering{2.4}  \\
    Anticipation Game Low Probability Group & \centering{2} & \centering{3.6}  \\
    \midrule
    Scarcity Game & \centering{2} & \centering{4} \\
    \bottomrule
  \end{tabular}
\end{table}

