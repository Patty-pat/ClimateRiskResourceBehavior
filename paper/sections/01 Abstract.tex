This study investigates how the increased risk of climate shocks influences individual resource extraction behavior in a common pool resource context. Conducted at the BonnEconLab using OTree, we simulated a resource extraction game under two scenarios: high (80\%) and low (20\%) risk of climate shocks, against a no-risk baseline. Our analysis employed parametric and non-parametric tests alongside a survey to assess additional factors influencing behavior. Results revealed significant increases in extraction rates under high risk compared to both low risk and baseline. %, with slight influence from environmental attitudes.
This highlights a distinct behavioral shift towards greater extraction in anticipation of climate shocks, diverging from existing literature focused on incentivizing pro-environmental behavior.
Moreover, we found evidence that self-assessed environmental attitudes correlate to lower extraction rates but only in a no-shock scenario.
