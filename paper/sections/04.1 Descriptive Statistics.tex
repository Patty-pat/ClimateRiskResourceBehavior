%\documentclass[cprs.tex]{subfiles}
 A total of 72 subjects partook in this experiment, and was conducted in three sessions with 24 subjects participating in each of the 3 sessions held. From all the 72 subjects, they were randomly assigned into groups of three and subsequently were then assigned to different treatment groups by randomization defined in the game. The subject split are summarized as follow:

 \begin{table}[H]
  \centering
  \caption{Lab Subjects \& Group Treatment Split}\label{tab:2}
  \begin{tabular}{llccc}
    \toprule
    \multirow{2}{*}{\textbf{Baseline}} & \multicolumn{2}{c}{\textbf{Anticipation}} & \multicolumn{2}{c}{\textbf{Shock}} \\ \cline{4-5}
                                    &         & & No & Yes \\
    \midrule
    \multirow{3}{*}{N Subjects = 72}& \multicolumn{2}{l}{\textbf{Low Probability}} \\ \cline{2-5}
                                    &Subjects & 36& 27 & 9 \\
                                    &Group    & 12 & 9 & 3 \\
                                    \cline{2-5}
    \multirow{3}{*}{N Groups = 24}& \multicolumn{2}{l}{\textbf{High Probability}} \\ \cline{2-5}
                                    &Subjects & 36& 9 & 27 \\
                                    &Group    & 12 & 3 & 9 \\
    \bottomrule
    \multicolumn{5}{l}{\footnotesize Source: BonnEconLab Subjects.}
  \end{tabular}
\end{table}


  % To examine randomization when assigning groups to low and high probability in the anticipation part of the game, we employ two rank-sum tests - one on individual level and one on a group level. The rank-sum test on an individual level compares the behaviour of individuals later assigned to a high probability against individuals who were later assigned to a low probability (\autoref{tab:individual_round1_hilo_ranksum_test}). For this test, we only use data from the first round of the baseline part, where there was no prior interaction among participants, thus providing a clear view of initial decision-making patterns.
%The rank-sum test on a group level compares the group mean of extraction rates over periods one to five in the baseline between the two sub-groups (\autoref{tab:hilo_ranksum_test_results}). The results suggest no significant differences at both individual and group level, indicating successful randomization.

%\noindent
%\input{Tables/h1_ranksum_individual}
%\input{Tables/h1_ranksum_group}

%\vspace{15pt}
