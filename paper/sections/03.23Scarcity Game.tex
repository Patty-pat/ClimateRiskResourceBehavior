In the third part of our experiment, we investigated the extent to which an experienced climate shock in the past influenced our participants in their extraction behaviour. To observe this scarcity effect we employed a within subject design. The participants were asked to make extraction decision as part of a group of three similar to the previous games. The main difference here is that in the Anticipation game, participants assigned to high or low probability shock groups may or may not experience shock at the end of the 5$^{th}$ round. The key treatment in the game is that, some participants did not experience climate shock, yet some did (see Table \ref{tab:2} for subject and group treatment allocation).

\noindent The purpose of this game is thus to see whether participants (and groups) with prior experience of scarcity due to forest fire will behave differently to those who have not experienced a climate shock before. To create a scarcity situation, participants were told that they will start this game with only 50 trees (50\% of baseline initial forest stock to mimic the condition set in the Anticipation game) instead of 100 trees. Additionally, the individual harvest restriction per round now depended on the number of trees in the forest as presented in Table \ref{tab:1}. Although each group began with a shared forest stock of 50 trees, group's shared forest may regrow beyond the initial 50 trees up to the maximum 100 trees. Thus, should the group chose to refrain from extracting beyond the growth rate of 10\% - that is less than 5 trees), the forest stock can regrow almost to its full capacity of 100 trees. Given that there is no uncertainty of another shock to occur, there is an incentive for participants in a group to maximize their individual payoff by maximizing the number or their shared forest resource.

\begin{table}[htbp!]
  \centering
  \caption{Maximum harvest table.}\label{tab:1}
  \begin{tabular}{ll}
    \toprule
    Number of trees in forest resource & Maximum allowable harvest \\
    \midrule
    21-100 & 7 \\
    18-20 & 6  \\
    15-17 & 5 \\
    12-14 & 4 \\
    9-11 & 3 \\
    6-8 & 2 \\
    3-5 & 1 \\
    0-2 & 0 \\
    \bottomrule
  \end{tabular}
\end{table}


\paragraph{\textit{The Payoff Function.}} The payoff function of individual and group under our scarcity game remain the same as in equation \eqref{eq:1}. The only difference if the number of starting trees in round 1 which affects the number of remaining trees left in the forest which in turn affects social payoff. Thus, we can reiterate $Z_{5}$ as
\begin{equation}
\label{eq:8}
    Z_{5} = \left( \left( \left( \left( \left(50 - X_{1}\right) 1.1 - X_{2}\right) 1.1 - X_{3} \right) 1.1 - X_{4} \right) 1.1 - X_{5} \right) 1.1
\end{equation}
%where it simplifies to
%\begin{equation}
%\label{eq:9}
%    Z_{5} = 80.5255 - 1.61051X_{1} - 1.4641X_{2} - 1.331X_{3} - 1.21X_{4} - 1.1X_{5}
%\end{equation}

\noindent Although the definition of $Z_5$ changes under scarcity game, the payoff function for individual and group remains the same as in the ones under baseline game defined in equation \ref{eq:2} for individual payoff function and \ref{eq:3} for group payoff function.

\noindent We re-write individual payoff function again below:
\begin{equation*}
    \pi_{i}=2\sum_{t=1}^{5} (x_{i,t})+\left( \frac{Z_{5} \times 4}{3} \right)
\end{equation*}

\noindent And the group's payoff function:
\begin{equation*}
    \Pi_{group} = \sum_{i=1}^{3} \pi_{i} = 2 \sum_{t=1}^{5} X_{t} + \left(Z_5 \times 4 \right)
\end{equation*}
