We conducted our lab experiment on January 2024 at the BonnEconLab to study the effect of climate shock on human extraction behaviour, particularly in the context of common pool resource. The experiment took place in three separate sessions, in each session 24 carefully selected participants took part in our lab experiment. Participants were chosen based on specific criteria: they had to be registered members of the BonnEconLab, proficient in German, and aged between 18 and 30 years. A crucial aspect of the selection process was also ensuring that these participants had not previously taken part in similar experiments. Since our study involves study with human subjects, our study has been pre-registered and approved for ethical review by the Institutional Review Board of the University of Bonn (Gesellschaft f\"ur experimentelle Wirtschaftsforschung (GfeW)) prior to conducting this experiment.

 Each of the three experimental session we conducted lasted approximately 45 minutes, and was divided into two main segments: a game followed by a comprehensive questionnaire, details of which are provided in the subsequent section. The design of our game segment follows the design of the forestry game by \citet{janssen2013breaking}.To build our instrument, we used O-tree, a Python-based software for experimental economic research\citet{chen2016otree}. No deception were done in our experiment, because complete instructions of the game were fully communicated through on screen display in the lab (see Appendix %\ref{app:A}).

 The game was divided into three parts: the baseline game as the control group, the anticipation treatments and the scarcity treatment. One part lasted five rounds, so the game lasted 15 rounds in total. Each participant played each part and was told in advance how many rounds the parts would last. Before the experiment began, each participant drew a number from 1 to 24 face down when they registered. This number indicated the participant’s cabin number for the experiment. We assigned the participants to groups of three depending on their cabin number, i.e. the participants with the numbers one to three were one group, the participants with the numbers four to six were one group and so on up to
24. During the experiment the participants were not permitted to communicate with each other, and
they were unaware of the group members to whom they had been assigned. The group assignments
remained fixed throughout the experiment.

All participants in each session of our study began by participating first in our Baseline game, followed by Anticipation game, Scarcity game and ended the experiment with a survey in which the order remain the same for all participants. Furthermore, in each game parts before participants began round 1 of each game, they were given a complete on screen instructions and subsequently asked to complete comprehension questions to ensure that participants fully understood the given instructions.
